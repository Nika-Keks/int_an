\documentclass[a4paper,12pt]{article}

\usepackage[hidelinks]{hyperref}
\usepackage{amsmath}
\usepackage{mathtools}
\usepackage{shorttoc}
\usepackage{cmap}
\usepackage[T2A]{fontenc}
\usepackage[utf8]{inputenc}
\usepackage[english, russian]{babel}
\usepackage{xcolor}
\usepackage{graphicx}
\usepackage{float}
\graphicspath{{./../../images/}}

\definecolor{linkcolor}{HTML}{000000}
\definecolor{urlcolor}{HTML}{0085FF}
\hypersetup{pdfstartview=FitH,  linkcolor=linkcolor,urlcolor=urlcolor, colorlinks=true}

\DeclarePairedDelimiter{\floor}{\lfloor}{\rfloor}

\renewcommand*\contentsname{Содержание}

\newcommand{\plot}[3]{
    \begin{figure}[H]
        \includegraphics[scale=0.75]{#1}
        \caption{#2}
        \label{#3}
    \end{figure}
}

\begin{document}
    
\begin{titlepage}
    \begin{center}
        \textsc{
            Санкт-Петербургский политехнический университет имени Петра Великого \\[5mm]
            Институт прикладной математики и механики\\[2mm]
            Кафедра прикладной математики
        }   
        \vfill
        \textbf{\large
            Интервальный анализ\\
            Отчёт по лабораторной работе №1 \\[3mm]
            %по курсовой работе \\[3mm]
        }                
    \end{center}

    \vfill
    \hfill
    \begin{minipage}{0.5\textwidth}
        Выполнил: \\[2mm]   
		Студент: Аникин Александр \\
		Группа: 3630102/80201\\
    \end{minipage}

	\hfill
	\begin{minipage}{0.5\textwidth}
		Принял: \\[2mm]
		к. ф.-м. н., доцент \\   
		Баженов Александр Николаевич
	\end{minipage}

    \vfill
    \begin{center}
        2022 г.
    \end{center}
\end{titlepage}
    \newpage

    \tableofcontents
    \newpage

    \listoffigures
    \newpage

    \listoftables
    \newpage

    \section{Постановка задачи}
    Для набора данных с интервальной неопределённостью построить нелинейную регрессию методом центра неопределённости и методом распознающего функционала.
    Сравнить результаты.

    \section{Теория}
    Пусть величина $ y $ является функцией от независимых аргументов $ x_{1}, x_{2}, ..., x_{p} $.
    \begin{equation}
        y = f(x, \beta)
        \label{eq:general}
    \end{equation}
    где $ x = (x_1, ..., x_n) $ - вектор независимых переменных, $ \beta = (\beta_{1}, ..., \beta_{k}) $ - вектор параметров функции.
    Имеется набор $ x = {x_{1}, ..., x_{m}}, y = {y_{1}, ..., y_{m}} $, необходимо найти оценку вектора параметров $ \beta $, который соответствует конкретной функции из семейства \ref{eq:general}.
    Будем рассматривать задачу, в которой $ x $ - точечный вектор, а $ y $ - интервальный вектор.
    \newline

    \noindent Будем рассматривать функцию вида:
    \begin{equation}
        y = \beta_{0} + \beta_{1} \cdot x + \beta_{2} \cdot x^{2}
    \end{equation}

    \subsection{Метод центра неопределённости}
    Для удобства отсортируем все входные параметры по $ x $ по возрастанию, и сместим значения так, чтобы $ x_{0} = 0 $.
    $ x_{i} - \text{число}, y_{i} = [h_{Hi}, h_{Bi}] - \text{интервал} \ \forall i = 1, ..., m $.

    \noindent\textbf{Вычисление коэффициента $ \beta_{0} $}
    \newline
    Рассмотрим тройку наблюдений $ y_{i}, y_{j}, y_{k} $ с упорядоченными значениями $ x_{i} < x_{j} < x_{k} $.
    Для каждой тройки индексов $ i < j < k $ вычислим допустимые значения коэффициента $ \beta_{0} $:
    \begin{equation}
        \beta_{0H0jk} = h_{H0}, \text{если} \ i = 0
    \end{equation}
    \begin{equation}
        \beta_{0Hijk} = \frac{x_{i}x_{j}x_{k}}{x_{j} - x_{i}} (\frac{h_{Hi}}{x_{i}(x_{k} - x_{i})} - \frac{h_{Bj}}{x_{j}(x_{k} - x_{j})} + \frac{h_{Hk}(x_{j} - x_{i})}{x_{k}(x_{k} - x_{i})(x_{k} - x_{j})})
    \end{equation}
    \begin{equation}
        \beta_{0B0jk} = h_{B0}, \text{если} \ i = 0
    \end{equation}
    \begin{equation}
        \beta_{0Bijk} = \frac{x_{i}x_{j}x_{k}}{x_{j} - x_{i}} (\frac{h_{Bi}}{x_{i}(x_{k} - x_{i})} - \frac{h_{Hj}}{x_{j}(x_{k} - x_{j})} + \frac{h_{Bk}(x_{j} - x_{i})}{x_{k}(x_{k} - x_{i})(x_{k} - x_{j})})
    \end{equation}
    Далее вычисляем:
    \begin{equation}
        \beta_{0min} = \max_{ijk}(\beta_{0Hijk})
    \end{equation}
    \begin{equation}
        \beta_{0max} = \min_{ijk}(\beta_{0Bijk})
    \end{equation}

    \noindent\textbf{Вычисление коэффициентов $ \beta_{1} $}
    \newline
    Аналогично вычислению $ \beta_{0} $ вычислим допустимые значения коэффициента $ \beta_{1} $:
    \begin{equation}
        \beta_{1Hijk} = \frac{(x_{i} + x_{k})(x_{j} + x_{k})}{x_{j} - x_{i}} (-\frac{h_{Bi}}{x_{k}^{2} - x_{i}^{2}} + \frac{h_{Hj}}{(x_{k}^{2} - x_{j}^{2})} - \frac{h_{Bk}(x_{j}^{2} - x_{i}^{2})}{(x_{k}^{2} - x_{j}^{2})(x_{k}^{2} - x_{i}^{2})})
    \end{equation}
    \begin{equation}
        \beta_{1Bijk} = \frac{(x_{i} + x_{k})(x_{j} + x_{k})}{x_{j} - x_{i}} (-\frac{h_{Hi}}{x_{k}^{2} - x_{i}^{2}} + \frac{h_{Bj}}{(x_{k}^{2} - x_{j}^{2})} - \frac{h_{Hk}(x_{j}^{2} - x_{i}^{2})}{(x_{k}^{2} - x_{j}^{2})(x_{k}^{2} - x_{i}^{2})})
    \end{equation}
    Далее вычислим:
    \begin{equation}
        \beta_{1min} = \max_{ijk}(\beta_{1Hijk})
    \end{equation}
    \begin{equation}
        \beta_{1max} = \min_{ijk}(\beta_{1Bijk})
    \end{equation}

    \noindent\textbf{Вычисление коэффициента $ \beta_{2} $}
    \newline
    Аналогично вычислению $ \beta_{0} $ и $ \beta_{1} $ вычислим допустимые значения коэффициета $ \beta_{2} $:
    \begin{equation}
        \beta_{2Hijk} = \frac{ \frac{h_{Hi}}{(x_{k} - x_{j})} - \frac{x_{Bj}}{(x_{k} - x_{j})} + \frac{h_{Hk}(x_{j} - x_{i})}{(x_{k} - x_{j})(x_{k} - x_{i})} }{(x_{j} - x_{i})}
    \end{equation}
    \begin{equation}
        \beta_{2Bijk} = \frac{ \frac{h_{Bi}}{(x_{k} - x_{j})} - \frac{x_{Hj}}{(x_{k} - x_{j})} + \frac{h_{Bk}(x_{j} - x_{i})}{(x_{k} - x_{j})(x_{k} - x_{i})} }{(x_{j} - x_{i})}
    \end{equation}
    Далее вычислим:
    \begin{equation}
        \beta_{2min} = \max_{ijk}(\beta_{2Hijk})
    \end{equation}
    \begin{equation}
        \beta_{2max} = \min_{ijk}(\beta_{2Bijk})
    \end{equation}

    \noindent\textbf{Оценка параметров зависимости}
    Далее для всех пар $ \beta_{imin}, \beta_{imax}, i = 1, 2, 3 $ проведём следующее сравнение.
    \begin{itemize}
        \item Если $ \beta_{imin} \leq \beta_{imax} $, то выборка совместна и значение коэффициента $ \beta_{i} $ лежит в интервале $ [\beta_{imin}, \beta_{imax}] $.
        \item Иначе выборка несовместна.
    \end{itemize}
    И в качестве точечной оценки коэффициента $ \beta_{i} $ возьмём $ \hat{\beta_{i}} = \frac{\beta_{imin} + \beta_{imax}}{2} $.
    \newline

    \subsection{Метод распознающего функционала}
    Перепишем исходную задачу в матричном виде:
    \begin{equation}
        X\beta = y
    \end{equation}
    где $ x, y $ - исходный набор, $ \beta $ - искомый вектор параметров,
    \begin{equation}
        X = 
        \begin{pmatrix}
            1 & x_{1} & x_{1}^2 \\
            1 & x_{2} & x_{2}^2 \\
            ... & ... & ... \\
            1 & x_{m} & x_{m}^2
        \end{pmatrix}
    \end{equation}
    \newline
    Введём распознающий функционал $ \text{Tol}(\beta, X, y) $.
    \begin{equation}
        \text{Tol}(\beta, X, y) = \min_{1 \leq i \leq m}(\text{rad}y_{i} - | \text{mid}y_{i} - \sum_{j = 1}^{n}X_{ij}\beta_{j} |)
    \end{equation}
    \newline
    Тогда оценкой ветора параметров будет $ \arg\max_{\beta}\text{Tol}(\beta, X, y) $
    
    \section{Реализация}
    Язык программирования: Python. Среда разработки: Visual Studio Code.

    \section{Результаты}
    \subsection{Генерация данных}
    Рассмотрим модель \ref{eq:general}. Для заданного набора входных значений $ x = {x_{1}, ..., x_{m}} $ посчитаем точечный вектор выходных значений $ y = {y_{1}, ..., y_{m}}$.
    Затем построим интервальный вектор выходных значений следующим образом: вместо каждого точечного значения $ y_{i} $ возьмём интервал $ [y_{i} - |\delta_{i1}|, y_{i} + |\delta_{i2}|] $,
    где $ \delta_{i1}, \delta_{i2} $ - случайные величины, $ \delta_{i1}, \delta_{i2} \in N(0, 5), \forall i = 1, ..., m $.

    \subsection{Результаты}
    Рассмотрим две модели:
    \begin{equation}
        y = 7 + 5x - \frac{1}{10}x^{2}
        \label{eq:task1}
    \end{equation}
    \noindent И
    \begin{equation}
        y = -42 - 17x + 5x^{2}
        \label{eq:task2}
    \end{equation}

    \noindent\textbf{Модель \ref{eq:task1}} \newline
    Сначала сгенерируем совместную выборку для модели \ref{eq:task1} размера $ 25 $. Исходный набор данных на рисунке ниже.
    \plot{ValidData25}{Совместная выборка для модели \ref{eq:task1}}{p:validdata25}
    
    \noindent Метод центра неопределённости дал следующие результаты:
    \begin{table}[H]
        \begin{center}
            \begin{tabular}{| c | c | c | c |}
                \hline
                & $ \beta_{imin} $ & $ \beta_{imax} $ & $ \hat{\beta} $ \\
                \hline
                $ \beta_{0} $ & 3.461 & 7.926 & 5.694 \\
                \hline
                $ \beta_{1} $ & 4.789 & 5.441 & 5.115 \\
                \hline
                $ \beta_{2} $ & -0.113 & -0.091 & -0.102 \\
                \hline
            \end{tabular}
        \end{center}
        \caption{Оценки коэффициентов зависимости методом центра неопределённости для выборки по модели \ref{eq:task1}}
        \label{t:validdata25}
    \end{table}

    \noindent Тогда найденная зависимость будет иметь следующий вид:
    \plot{UndefinedcenterValidData25}{Метод центра неопределённости для выборки по модели \ref{eq:task1}}{p:ucvaliddata25}
    \plot{CorridorUndefinedCenterValidData25}{Коридор совместных зависимостей}{p:corridorvaliddata25}

    \noindent Для наглядности также рассмотрим выборку на том же промежутке, но с большим числом элементов ($ 150 $), и посмотрим на полеченный коридор совместных зависимостей.
    \plot{CorridorUndefinedCenterValidData150}{Коридор совместных зависимостей для выборки большего размера}{p:corridorvaliddata150}

    \noindent В свою очередь метод распознающего функционала дал следующую оценку коэффициентов:
    $ \hat{\beta_{0}} = 7.376, \hat{\beta_{1}} = 4.950, \hat{\beta_{2}} = -0.099 $.
    Причём значение распознающего функционала на этом наборе положительно: $ \text{Tol}(\hat{\beta}) = 0.260 $.
    \newline
    И график построенной зависимости:
    \plot{TolValidData25}{Метод распознающего функционала для выборки по модели \ref{eq:task1}}{p:tolvaliddata25}

    \noindent Теперь внесём пять случайных малых изменений в исходную выборку, сделав её несовместной.
    Новая выборка будет иметь следующий вид:
    \plot{DataWithEstims25}{Несовместная выборка для модели \ref{eq:task1}}{p:estimdata25} 
    
    \noindent Методом центра неопределённости получим следующие результаты:
    \begin{table}[H]
        \begin{center}
            \begin{tabular}{| c | c | c | c |}
                \hline
                & $ \beta_{imin} $ & $ \beta_{imax} $ & $ \hat{\beta} $ \\
                \hline
                $ \beta_{0} $ & 471.345 & -780.171 & - \\
                \hline
                $ \beta_{1} $ & 105.733 & -46.105 & - \\
                \hline
                $ \beta_{2} $ & 1.300 & -3.301 & - \\
                \hline
            \end{tabular}
        \end{center}
        \caption{Оценки коэффициентов зависимости методом центра неопределённости для несовместной выборки по модели \ref{eq:task1}}
        \label{t:estimdata25}
    \end{table}
    \noindent Тем самым выборка несовместна и метод центра неопределённости даёт некорректные значения коэффициентов. 

    \noindent В свою очередь метод распознающего функционала дал следующую оценку коэффициентов:
    $ \hat{\beta_{0}} = 1.366, \hat{\beta_{1}} = 6.394, \hat{\beta_{2}} = -0.155 $.
    Но в этом случае значение распознающего функционала на этом наборе меньше нуля: $ \text{Tol}(\hat{\beta}) = -2.548 $.
    \newline
    И график построенной зависимости:
    \plot{TolDataWithEstims25}{Метод распознающего функционала для несовместной выборки по модели \ref{eq:task1}}{p:tolestimdata25}

    \noindent\textbf{Модель \ref{eq:task2}} \newline
    Теперь сгенерируем совместную выборку для модели \ref{eq:task2} размера $ 50 $ и проделаем с ней аналогичную работу.
    Исходная выборка имеет следующий вид:
    \plot{ValidData50}{Совместная выборка для модели \ref{eq:task2}}{p:validdata50}

    \noindent Метод центра неопределённости дал следующие результаты:
    \begin{table}[H]
        \begin{center}
            \begin{tabular}{| c | c | c | c |}
                \hline
                & $ \beta_{imin} $ & $ \beta_{imax} $ & $ \hat{\beta} $ \\
                \hline
                $ \beta_{0} $ & -42.438 & -41.488 & -41.963 \\
                \hline
                $ \beta_{1} $ & -17.374 & -16.768 & -17.071 \\
                \hline
                $ \beta_{2} $ & 4.972 & 5.047 & 5.009 \\
                \hline
            \end{tabular}
        \end{center}
        \caption{Оценки коэффициентов зависимости методом центра неопределённости для выборки по модели \ref{eq:task2}}
        \label{t:validdata50}
    \end{table}

    \noindent Тогда график построенной зависимости будет иметь следующий вид:
    \plot{UndefinedCenterValidData50}{Метод центра неопределённости для выборки по модели \ref{eq:task2}}{p:ucvaliddata50}
    \plot{CorridorUndefinedCenterValidData50}{Коридор совместных зависимостей}{p:corridorvaliddata50}

    \noindent Метод распознающего функционала для следующую оценку коэффициентов зависимости:
    $ \hat{\beta_{0}} = -10.0, \hat{\beta_{1}} = -10.0, \hat{\beta_{2}} = 2.425 $.
    Причём значение распознающего функционала на этом значении меньше нуля: $ \text{Tol}(\hat{\beta}) = -36.398 $.
    
    \noindent И тогда график найденной зависимости будет иметь следующий вид:
    \plot{TolValidData50}{Метод распознающего функционала для выборки по модели \ref{eq:task2}}{p:tolvaliddata50}

    \noindent Теперь внесём пять малых случайных изменений в выборку так, чтобы она стала несовместной. Тогда получим выборку:
    \plot{DataWithEstims50}{Несовместная выборка для модели \ref{eq:task2}}{p:estimdata50}

    \noindent Метод центра неопределённости дал следующие результаты:
    \begin{table}[H]
        \begin{center}
            \begin{tabular}{| c | c | c | c |}
                \hline
                & $ \beta_{imin} $ & $ \beta_{imax} $ & $ \hat{\beta} $ \\
                \hline
                $ \beta_{0} $ & 584.565 & -1575.076 & - \\
                \hline
                $ \beta_{1} $ & 719.263 & -292.551 & - \\
                \hline
                $ \beta_{2} $ & 35.251 & -83.229 & - \\
                \hline
            \end{tabular}
        \end{center}
        \caption{Оценки коэффициентов зависимости методом центра неопределённости для несовместной выборки по модели \ref{eq:task2}}
        \label{t:estimdata50}
    \end{table}

    \noindent А в свою очередь метод распзнающего функционала дал следующую оценку коэффициентов:
    $ \hat{\beta_0} = -5.886, \hat{\beta_{1}} = -10.0, \hat{\beta_{2}} = 2.257 $.
    Распознающий функционал на этом наборе также имеет отрицательное значение: $ \text{Tol}(\hat{\beta}) = -40.183 $.

    \noindent График построенной зависимости имеет следующий вид:
    \plot{TolDataWithEstims50}{Метод распознающего функционала для несовместной выборки по модели \ref{eq:task2}}{p:tolestimdata50}

    \section{Обсуждение}
    Из полученных результатов можно сделать следующие наблюдения.
    Для совместной выборки метод центра неопределённости более точно оценивает коэффициенты параметров зависимости.
    Особенно заметно лучшие оценки коэффициентов метод центра неопределённости строит для второй выборки, что видно на рисунках ~\ref{p:ucvaliddata25}, ~\ref{p:ucvaliddata50}, ~\ref{p:tolvaliddata25}, ~\ref{p:tolvaliddata50}.
    Причём для обеих выборок построенная методом центра неопределённости зависимость проходит через все интервалы, что видно на рисунках ~\ref{p:ucvaliddata25}, ~\ref{p:ucvaliddata50}.
    \newline
    Но в свою очередь метод центра неопределённости оказывается сильно чувствительным к выбросам, что можно заметить из результатов в таблицах ~\ref{t:estimdata25}, ~\ref{t:estimdata50}.
    Даже при совсем небольших отклонениях от совместных данных метод центра неопределённости даёт некорректные результаты.
    Метод распознающего функционала является куда менее чувствительным к выбросам, что видно на рисунках ~\ref{p:tolvaliddata25}, ~\ref{p:tolestimdata25},
    хотя всё же на выборке с выбросами максимальное значение распознающего функционала становиться отрицательным, что также говорит о несовместности системы.
    \newline
    Также стоит отметить, что даже на совместной выборке метод распознающего функционала может давать плохие результаты, что видно на рисунке ~\ref{p:tolvaliddata50}.
    Причём максимальное значение распознающего функционала в этом случае сильно меньше нуля.  
    \newline
    Из рисунков ~\ref{p:corridorvaliddata25}, ~\ref{p:corridorvaliddata50}, ~\ref{p:corridorvaliddata150} можно сделать предположение, что ширина коридора зависит от числа элементов в выборке.
    Это особенно наглядно видно, на рисунках ~\ref{p:corridorvaliddata25}, ~\ref{p:corridorvaliddata150}.
    \newline
    Также стоит отметить, что метод центра неопределённости даёт наименьшую неопределённость для коэффициента $ \beta_{2} $, а наибольшую для коэффициента $ \beta_{0} $, что видно на таблицах ~\ref{t:validdata25} - ~\ref{t:estimdata50}.

\end{document}